\subsection{Use case 4}
I denne use case vil brugeren indtaste volumen på karet for at systemet kan vide hvor meget gødning der skal doseres. Der kan også ændres på volumen i perioder hvor der bruges lidt vand for at undgå vandet bliver dårligt. For at use casen kan gennemføres skal teknikeren have oprettet et kar som er tilkoblet systemet.
\begin{usecase}

\addtitle{Use Case 4}{Indtast volumen} 

\additemizedfield{Mål:}{
\item Indtaste volumen på et vandkar oprettet i systemet
}

\addfield{Initieret af:}{
	Bruger   
}

\additemizedfield{Aktører:}{\item Primær: Bruger
							\item Sekundær: Tekniker}

\addfield{Samtidige forekomster:}{1}

%Preconditions: What must be true on start and worth telling the reader?
\addfield{Prækondition:}{Der er oprettet et kar i systemet og det er tilkoblet}

%Postconditions: What must be true on successful completion and worth telling the reader
\addfield{Postkondition:}{Systemet er opdateret med volumen på karet }

%Main Success Scenario: A typical, unconditional happy path scenario of success.
\addscenario{Hovedscenarie:}{
	\item Bruger trykker på "Kar 1" på GUI 
	\begin{itemize}
	\item Systemet viser en menu hvor det er muligt at
	      indtaste volumen, adresse og pH-værdi i karet
	      samt slette det 
	\end{itemize}	 
	\item Bruger trykker på "Volumen"
	\begin{itemize}
	\item Systemet viser en cursor i skrivefeltet tilhørende
		  volumen
	\end{itemize}
	\item Bruger indtaster en volumen i enheden Liter
	\begin{itemize}
	\item Systemet viser den nye værdi 
	\end{itemize}	
	\item Bruger trykker på "Retuner"
	\begin{itemize}
	\item Systemet opdater volumen
	\end{itemize}
	\begin{itemize}
	\item Systemet retunerer til hovedmenuen
	\end{itemize}	 
}

%Extensions: Alternate scenarios of success or failure.
%\addscenario{Udvidelser:}{
%	\item[Ex.1] Teknikeren ønsker kun at aflæse værdier:
%		\begin{enumerate}
%		\item[1.] Teknikeren trykker på "OK"
%		\end{enumerate}
%}


\end{usecase}
