
%%%%%%%%%%%%%%%%%%%%%%%Opsætning af format%%%%%%%%%%%%%%%%%%%%%%%%%%
\documentclass[a4paper,oneside]{memoir} %A4papir, to side, størrelse 12, type memoir 

% For dansk opsætning med æ, ø og å, samt pænere orddeling.
\usepackage[utf8]{inputenc}				% æøå
\usepackage[danish]{babel}				% dansk opsætning
\renewcommand{\danishhyphenmins}{22}	% fikser babel fejl/bedre orddeling
\usepackage[T1]{fontenc}
\usepackage{lmodern} 

%%Høre til under tabel (Pakker, men skal stå før pgfplot pga. default værdier)%%
\usepackage[table]{xcolor}				
%%%%%%%%%%%%%%%%%%%%%%%%%%%%%%%%%%%%%%%%%%%%%%%%%%%%%%%%%%%%%%%%%%%%%%%%%%%%%%%%

\setcounter{tocdepth}{4} % inkludere sub + subsubsection i inholdsfortegnelde
\setsecnumdepth{subsection}

\usepackage{titlesec} 
\titleformat{\chapter}		  % Fjerne Kapitel og tal fra chapters
			{\Large\bfseries} % format
			{}                % label
			{0pt}             % sep
			{\huge}           % before-code
% til mellemrum

\titlespacing\section{0pt}
{18pt plus 4pt minus 2pt}{6pt plus 2pt minus 2pt} %halvt mellemrum efter section
\titlespacing\subsection{0pt}
{12pt plus 4pt minus 2pt}{2pt plus 2pt minus 2pt} %kun lidt mellemrum efter subsection
\titlespacing\subsubsection{0pt}
{12pt plus 4pt minus 2pt}{2pt plus 2pt minus 2pt} %kun lidt mellemrum efter subsubsection 

\setlength\parindent{0pt} %Ingen indryk efter ny afsnit

%Marginer indstilles
\setlrmarginsandblock{3cm}{3cm}{*}		%Højre - venstre
\setulmarginsandblock{3cm}{2.5cm}{*}	%Øverst - nederst
\checkandfixthelayout[nearest]    		%Specifikt valg af højde algoritme
\usepackage{ragged2e,anyfontsize}		% Justering af elementer
\usepackage{fixltx2e}					% Retter forskellige fejl i LaTeX-kernen

% Sidehoved og -fod
\let\footruleskip\undefined  %fixer memoir default footruleskip
\usepackage{fancyhdr}
\pagestyle{fancy}
\fancyhf{}
\fancyhead[C]{\textit{Automatisk Vandingssystem}}
\fancyfoot[LO,RE]{\thepage\ af \thelastpage}  	% sættet sidetal tal h/v efter om der er ulige
												% eller lige sidetal
\fancypagestyle{plain}{% bruges ved Undtagelser						
  \fancyhf{}%
  \fancyfoot[RO,LE]{\thepage\ af \thelastpage}	
  \renewcommand{\headrulewidth}{0pt}			%Ingen linje ved chapter, kun sidetal
}
%%%%%%%%%%%%%%%%%%%%%%%%%%%%%%%%%%%%%%%%%%%%%%%%%%%%%%%%%%%%%%%%%%%

%%%%%%%%%%%%%%%%%%%%Pakker og design til forside%%%%%%%%%%%%%%%%%%%
\usepackage{amsmath}
\usepackage{tikz}
\usepackage{epigraph}


\renewcommand\epigraphflush{flushright}
\renewcommand\epigraphsize{\normalsize}
\setlength\epigraphwidth{1\textwidth}

\definecolor{titlepagecolor}{cmyk}{1,.60,0,.40}

\DeclareFixedFont{\titlefont}{T1}{ppl}{b}{it}{0.5in}




% The following code is borrowed from: http://tex.stackexchange.com/a/86310/10898

\newcommand\titlepagedecoration{%
\begin{tikzpicture}[remember picture,overlay,shorten >= -10pt]

\coordinate (aux1) at ([yshift=-15pt]current page.north east);
\coordinate (aux2) at ([yshift=-410pt]current page.north east);
\coordinate (aux3) at ([xshift=-4.5cm]current page.north east);
\coordinate (aux4) at ([yshift=-150pt]current page.north east);

\begin{scope}[titlepagecolor!40,line width=12pt,rounded corners=12pt]
\draw
  (aux1) -- coordinate (a)
  ++(225:5) --
  ++(-45:5.1) coordinate (b);
\draw[shorten <= -10pt]
  (aux3) --
  (a) --
  (aux1);
\draw[opacity=0.6,titlepagecolor,shorten <= -10pt]
  (b) --
  ++(225:2.2) --
  ++(-45:2.2);
\end{scope}
\draw[titlepagecolor,line width=8pt,rounded corners=8pt,shorten <= -10pt]
  (aux4) --
  ++(225:0.8) --
  ++(-45:0.8);
\begin{scope}[titlepagecolor!70,line width=6pt,rounded corners=8pt]
\draw[shorten <= -10pt]
  (aux2) --
  ++(225:3) coordinate[pos=0.45] (c) --
  ++(-45:3.1);
\draw
  (aux2) --
  (c) --
  ++(135:2.5) --
  ++(45:2.5) --
  ++(-45:2.5) coordinate[pos=0.3] (d);   
\draw 
  (d) -- +(45:1);
\end{scope}
\end{tikzpicture} }
%%%%%%%%%%%%%%%%%%%%%%%%%%%%%%%%%%%%%%%%%%%%%%%%%%%%%%%%%%%%%%%%%%%%

%%%%%%%%%%%%%%%%%%%%%%%%%%%%%%Pakker%%%%%%%%%%%%%%%%%%%%%%%%%%%%%%%%
\usepackage{graphicx} 				% Haandtering af eksterne billeder (JPG, PNG, EPS, PDF)
\usepackage{multirow}               % Fletning af raekker og kolonner (\multicolumn og \multirow)

\usepackage{wrapfig}				%for figure der skal have tekst om sig
\usepackage{float}					% Muliggoer eksakt placering af floats, f.eks. \begin{figure}[H]

\usepackage{enumerate}				% Muliggoer at man kan bruge fx a) i enumerate

\usepackage{pdflscape}				% Muligør landskab på enkelte sider
\usepackage{tabularx}				% Tabeller med X width

\usepackage{../Latex/Usecases/usecases}				% Usecase

\usepackage{mathtools}				% Formler og matematik

\usepackage{caption}
%\usepackage{subcaption}			% This package kills me :/

\usepackage{pdfpages}				% til at inkludere PDF filer

% For at holde styr på mangler i teksten
\usepackage[footnote,draft,danish,silent,nomargin]{fixme}	
%%%%%%%%%%%%%%%%%%%%%%%%%%%%%%%%%%%%%%%%%%%%%%%%%%%%%%%%%%%%%%%%%%%%

% ¤¤ Litteraturlisten ¤¤ %
\usepackage[danish]{varioref}				% Muliggoer bl.a. krydshenvisninger med sidetal (\vref)
\usepackage{natbib}							% Udvidelse med naturvidenskabelige citationsmodeller
\bibpunct[,]{[}{]}{;}{a}{,}{,} 				% Definerer de 6 parametre ved Harvard henvisning 
											% (bl.a. parantestype og seperatortegn)
\bibliographystyle{../Latex/Litteratur/harvard}		% Udseende af litteraturlisten.
\usepackage{hyperref}

\begin{document}
\listoffixmes
\section{Tabel eksempel}
Her en fancy tabel over noget pjat... jeg skal bare skrive noget text her

\begin{table}[H]
\centering
{\rowcolors{2}{white!70!black!60}{white!80!black!30} %farver på hver anden række -starter på 3
\setlength{\arrayrulewidth}{0.2mm}					 %tykkelse på linier 
\setlength{\tabcolsep}{10pt}						 %indryk i celle 
\renewcommand{\arraystretch}{1.5}					 %højden på tabelrum
\center
\begin{tabular}{ |p{2cm}|p{2cm}|p{2cm}|p{5cm}|}		 %længden på alle rum
\hline
\rowcolor{white!20!black!90}
\textcolor{white}{\large{\textbf{Name}}} &	
\textcolor{white}{\large{\textbf{Input}}}&	
\textcolor{white}{\large{\textbf{Output}}}&
\textcolor{white}{\large{\textbf{Description}}}\\
\hline
ting1	 	& AF	 	& AFG 	& X\\
ting2  	 	& AX   		& ALA 	& X\\
ting3 	 	& AL 		& ALB 	& X\\
ting4   	& DZ 		& DZA 	& X\\
ting5 		& AS		& ASM 	& X\\
ting6 		& AD 		& AND   & X\\
ting7 		& AO 		& AGO 	& X\\
\hline
\end{tabular}
}
\caption{Table to test captions and labels}
\label{table:eksempelf}
\end{table}

Og her er noget text til tabel \ref{table:eksempelf} $\leftarrow$ der blev lavet en reference.

\begin{table}[H]
\centering
\begin{tabular}{ | l | l | l | l | l | l | l | l |}
\hline
text & text & text & text & text & text & text & text \\
\hline
text & text & text & text & text & text & text & text \\
\hline
text & text & text & text & text & text & text & text \\
\hline  
\end{tabular}
\caption{Tabel}
\label{table:eksempels}
\end{table}

tabel \ref{table:eksempels} er en simpel tabel
\section{Usecase 1}
Beskrivelse af denne use case
\begin{usecase}

\addtitle{Use Case 1}{Template test} 

%Scope: the system under design
\addfield{Scope:}{System-wide}

%Level: "user-goal" or "subfunction"
\addfield{Level:}{User-goal}

%Primary Actor: Calls on the system to deliver its services.
\addfield{Primary Actor:}{End-User}

%Stakeholders and Interests: Who cares about this use case and what do they want?
\additemizedfield{Stakeholders and Interests:}{
	\item Stakeholder 1 name: his interests
	\item Stakeholder 2 name: his interests
}

%Preconditions: What must be true on start and worth telling the reader?
\addfield{Preconditions:}{}
%when multiple
%\additemizedfield{Preconditions:}{} 

%Postconditions: What must be true on successful completion and worth telling the reader
\addfield{Postconditions:}{}
%when multiple
%\additemizedfield{Preconditions:}{}

%Main Success Scenario: A typical, unconditional happy path scenario of success.
\addscenario{Main Success Scenario:}{
	\item The first action
	\item The second action
}

%Extensions: Alternate scenarios of success or failure.
\addscenario{Extensions:}{
	\item[2.a] Invalid login data:
		\begin{enumerate}
		\item[1.] System shows failure message
		\item[2.] User returns to step 1
		\end{enumerate}
	\item[5.a] Invalid subsriber data:
		\begin{enumerate}
		\item[1.] System shows failure message
		\item[2.] User returns to step 2 and corrects the errors
		\end{enumerate}
}

%Special Requirements: Related non-functional requirements.
\additemizedfield{Special Requirements:}{
	\item first applicable non-functional requirement
	\item second applicable non-functional requirement
}

%Technology and Data Variations List: Varying I/O methods and data formats.
\addscenario{Technology and Data Variations List:}{
	\item[1a.] Alternative first action with other technology
}

%Frequency of Occurrence: Influences investigation, testing and timing of implementation.
\addfield{Frequency of Occurrence:}{}

%Miscellaneous: Such as open issues/questions
%\addfield{Open Issues:}{}

\end{usecase}



\section{fodnote + reference/litteratur + fixme}
Når du laver en fodnote skriver man\footnote{Det der skal stå i fodnoten} (se texfilen)
\newline\newline
når der laves en reference skrives den i litteratur, derefter kan refernecen laves sådan,\footnote{\citep[ s. 13]{mti:AN236}}
\newline\newline
Det kan være der er noget tekst der ikke er godt nok, så kan men lave en fixmenote til de andre bruger, så står den i en liste på først side over fixmenotes\fxnote{det her skal uddybes!}

\begingroup
	\raggedright
	\nocite{*}
	\bibliography{Litteratur/litteratur}
\endgroup
\end{document}
